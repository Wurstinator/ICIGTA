\documentclass{article}

\usepackage{amsmath, amssymb, amsthm}
\usepackage[utf8]{inputenc}
\usepackage{geometry}
\newtheorem{theorem}{Theorem}
\newtheorem{definition}{Definition}

\begin{document}

\section{Cantor Space \& Borel Hierarchy}
\begin{definition}
	The \textbf{Cantor space} is the pair $(\mathbb{B}^\omega, d)$ with $d(\alpha, \beta) = \begin{cases} 0 & \text{if } \alpha = \beta \\ 2^{-(\min_n \alpha(n) \neq \beta(n))} & \text{else}  \end{cases}$. 
	
	Note: The $\frac{1}{2n}$-neighborhood of $\alpha$ is $\alpha[0, n] \cdot \mathbb{B}^\omega$.
\end{definition}

\begin{definition}
	From the Cantor space we define the \textbf{Cantor topology} with open sets $\mathcal{O} = \{W \cdot \mathbb{B}^\omega \mid W \subseteq \mathbb{B}^*\}$.
\end{definition}

\begin{definition}
	The \textbf{Borel hierarchy} is a collection $\{ \Sigma_1, \Pi_1, \Sigma_2, \Pi_2, \dots \}$ defined as \\
	$\Sigma_1 = \mathcal{O}$ \\
	$\Pi_1 = \mathbb{B}^\omega \setminus \mathcal{O}$ \\
	$\Sigma_{n+1} = \{ \bigcup_{i \in \mathbb{N}} L_i \mid L_i \in \Pi_n \}$ \\
	$\Pi_{n+1} = \{ \bigcap_{i \in \mathbb{N}} L_i \mid L_i \in \Sigma_n \}$
\end{definition}

\begin{theorem}
\begin{enumerate}
	\item Every class $\Sigma_n$ or $\Pi_n$ of the Borel hierarchy is closed under finite union and intersection. 
	\item For every $L \subseteq \mathbb{B}^\omega$, we have $L \in \Sigma_n$ iff $L^\complement \in \Pi_n$.
\end{enumerate}
\end{theorem}
\begin{proof}
	\begin{enumerate}
		\item It suffices to show the closure for $\Sigma_n$. Then it follows for $\Pi_n$ from (2). For example, let $L, K \in \Pi_n$, so $L^\complement, K^\complement \in \Sigma_n$, so $L^\complement \cup K^\complement = (L \cap K)^\complement \in \Sigma_n$ and therefore $L \cap K \in \Pi_n$.
		
			For $n = 0$, this is clear from the definition of $\mathcal{O}$. Let $W_1, W_2 \subseteq \mathbb{B}^*$. Then $W_1 \cdot \mathbb{B}^\omega \cup W_2 \cdot \mathbb{B}^\omega = (W_1 \cup W_2) \cdot \mathbb{B}^\omega$ and $W_1 \cdot \mathbb{B}^\omega \cap W_2 \cdot \mathbb{B}^\omega = (W_1 \cdot \mathbb{B}^* \cap W_2 \cdot \mathbb{B}^*) \cdot \mathbb{B}^\omega$.
			
			Using an induction argument, consider $L, K \in \Sigma_{n+1}$, so $L = \bigcup\limits_{i \in \mathbb{N}} L_i$ and $K = \bigcup\limits_{i \in \mathbb{N}} K_i$ for $(L_i)_i, (K_i)_i \in \Pi_n$. By induction, $L_i \cup K_i \in \Pi_n$ for all $i$, and thus $L \cup K = \bigcup\limits_{i \in \mathbb{N}} (L_i \cup K_i) \in \Sigma_{n+1}$. For intersection we have $L \cap K = \bigcup\limits_{i,j \in \mathbb{N}} L_i \cap K_j \in \Sigma_{n+1}$.
			
		\item De Morgan law %TODO F24
	\end{enumerate}
\end{proof}

\begin{definition}
	Let $f : \mathbb{B}^\omega \rightarrow \mathbb{B}^\omega$. $f$ is \textbf{continuous} if for all open sets $O \in \mathcal{O}$: $f^{-1}(O) \in \mathcal{O}$. 
	
	For $L, K \subseteq \mathbb{B}^\omega$, we write $K \leq L$ if there is a continuous function $f$ with $f^{-1}(L) = K$.
\end{definition}

\begin{theorem}
	Let $f : \mathbb{B}^\omega \rightarrow \mathbb{B}^\omega$. The following three statements are equivalent:
	\begin{enumerate}
		\item $f$ is continuous.
		\item $\forall \alpha \in \mathbb{B}^\omega. \forall n \in \mathbb{N}. \exists m \in \mathbb{N}. \forall \beta \in \mathbb{B}^\omega : d(\alpha, \beta) < \frac{1}{2^m} \Rightarrow d(f(\alpha), f(\beta)) \leq \frac{1}{2^n}$
		\item $\forall n \in \mathbb{N}. \exists m \in \mathbb{N}. \forall \alpha, \beta \in \mathbb{B}^\omega : d(\alpha, \beta) < \frac{1}{2^m} \Rightarrow d(f(\alpha), f(\beta)) \leq \frac{1}{2^n}$
	\end{enumerate}
\end{theorem}
\begin{proof}
	$\boldsymbol{(1) \Rightarrow (3)}$ %TODO
	
	$\boldsymbol{(3) \Rightarrow (2)}$: Trivial, since $m$ does not depend on $\alpha$ in general.
	
	$\boldsymbol{(2) \Rightarrow (1)}$: Let $L = W \cdot \mathbb{B}^\omega \in \mathcal{O}$. Let $U = \{ u \in \mathbb{B}^* \mid f(u \cdot \mathbb{B}^\omega) \subseteq L \}$. We claim that $f^{-1}(L) = U \cdot \mathbb{B}^\omega$.
	
	Let $\alpha \in U \cdot \mathbb{B}^\omega$, so $\alpha = u \cdot \beta$ for some $u \in U$. By definition of $U$, $f(u \cdot \beta) = f(\alpha) \in L$. Therefore, $\alpha \in f^{-1}(L)$.
	
	Let $\alpha \in f^{-1}(L)$, so $f(\alpha) = w \alpha'$ for some $w \in W$. Using the assumption, there is an $m \in \mathbb{N}$ such that for all $\beta \in \mathbb{B}^\omega$ with $d(\alpha, \beta) \leq 2^{-m}$, we have $d(f(\alpha), f(\beta)) \leq 2^{-|w|}$, meaning that $w \sqsubseteq f(\beta)$, so $f(\beta) \in L$. \\
	For all $\beta \in \mathcal{B}^\omega$, we have $f(\alpha[0,m] \cdot \beta) \in L$ by the previous result. This means $\alpha[0,m] \in U$ (by definition of $U$) and therefore $\alpha \in U \cdot \mathcal{B}^\omega$.
\end{proof}

\begin{theorem}
	If $K \leq L$ and $L \in \Sigma_n$, then $K \in \Sigma_n$. The same is true for $\Pi_n$.
\end{theorem}
\begin{proof}
	Let $f$ be a continuous function with $f^{-1}(L) = K$. For $n = 0$, $L \in \mathcal{O}$, so $K \in \mathcal{O}$.
	
	Otherwise, assume the claim is ture for $n$ and let $L \in \Sigma_{n+1}$, so $L = \bigcup\limits_{i \in \mathbb{N}} L_i$ for $L_i \in \Pi_n$. Let $K_i = f^{-1}(L_i)$, so we have $K_i \leq L_i$. By induction this gives us $K_i \in \Pi_n$ for all $i$ and therefore $\bigcup\limits_{i \in \mathbb{N}} K_i \in \Sigma_{n+1}$. It remains to be shown that $K = \bigcup\limits_{i \in \mathbb{N}} K_i$. %TODO
\end{proof}

\begin{definition}
	Let $L \subseteq \mathbb{B}^\omega$. $L$ is \textbf{complete} for $\Sigma_n$ if $\forall K \in \Sigma_n: K \leq L$. 
\end{definition}


\subsection{Relation to Automata}
\begin{itemize}
	\item regular $\Sigma_1$ = E-recognizable
	\item regular $\Pi_1$ = A-recognizable
	\item regular $\Sigma_2$ = co-Büchi-recognizable
	\item regular $\Pi_2$ = DBA-recognizable
	\item boolean combination of $\Pi_2$ = NBA-recognizable
\end{itemize}


\section{Gale-Stewart \& Wadge}
\begin{definition}
	Let $L \subseteq \mathbb{B}^\omega$. The \textbf{Gale-Stewart game} $\Gamma(L)$ is defined as follows: Starting with player 0, two players alternatingly pick bits 0 or 1, resulting in a play $\alpha \in \mathbb{B}^\omega$. Player 0 wins iff $\alpha \in L$.
\end{definition}

\begin{definition}
	Let $K, L \subseteq \mathbb{B}^\omega$. The \textbf{Wadge game} $W(K, L)$ is defined as follows: Starting with player 0, two players alternatingly pick bits 0 or 1, where player 1 also has the option to skip a turn, resulting in a pair $(\alpha, \beta)$ with $\alpha \in \mathbb{B}^\omega$ and $\beta \in \mathbb{B}^* \cup \mathbb{B}^\omega$.
	
	Player 1 wins the play $(\alpha, \beta)$ iff $\beta \in \mathbb{B}^\omega$ and $\alpha \in K \leftrightarrow \beta \in L$.  
\end{definition}

\begin{theorem}[Gale-Stewart]
	For $L \in \Sigma_1 \cup \Pi_1$, $\Gamma(L)$ is determined.
\end{theorem}
\begin{proof}
	If $L \in \Sigma_1$, then $L = W \cdot \mathbb{B}^\omega$. A winning strategy for player 0 (if it exists) is the attractor strategy for $W$.
	
	If $L = (W \cdot \mathbb{B}^\omega)^\complement \in \Pi_1$, then player 1 can play the attractor strategy for $W$.
\end{proof}

\begin{theorem}[Martin]
	For every set $L$ in the Borel hierarchy, $\Gamma(L)$ is determined.
\end{theorem} 
% no proof

\begin{theorem}
	Let $K, L \subseteq \mathbb{B}^\omega$. Player 1 wins $W(K, L)$ iff $K \leq L$.
\end{theorem}
\begin{proof}
	$\boldsymbol{\Rightarrow}$ Let $\sigma : (\mathbb{B}^* \times \mathbb{B}^*) \rightarrow \{0, 1, \varepsilon\}$ be a winning strategy for player 1 in $W(K, L)$. Let $\tau(\alpha)$ be a strategy for player 0 in which they play $\alpha(i)$ in turn $i$, and for all $\alpha \in \mathbb{B}^\omega$ let $f(\alpha)$ be the play of player 1 if both players play according to $\tau(\alpha)$ and $\sigma$ respectively. We claim that $f$ is continuous. %TODO
	
	$\boldsymbol{\Leftarrow}$ Let $f : \mathbb{B}^\omega \rightarrow \mathbb{B}^\omega$ be continuous. %TODO
\end{proof}

\paragraph{Example} Let $L = (0^*1)^\omega$. We claim that $L$ is $\Pi_2$-complete. Let $K = \bigcap\limits_{i \in \mathbb{N}} K_i \in \Pi_2$ for $K_i \in \mathcal{O}$, so $K_i = W_i \cdot \mathbb{B}^\omega$. We define a winning strategy for player 1 in $W(K, L)$ which proves the claim. \\
	At the beginning of the game, set a variable $i := 0$. In each turn, let $(u, v)$ be the play up until this point. If $u \notin W_i$, play 0. Otherwise, play 1 and increment $i$ by 1.
	
\begin{theorem}
	There is a language $L \subseteq \mathbb{B}^\omega$ such that $\Gamma(L)$ is not determined.
\end{theorem}
\begin{proof}
	Let On be the set of ordinal numbers. Let $\mathcal{S}_0$ and $\mathcal{S}_1$ be the set of strategies for player 0 and 1 respectively in a Gale-Stewart game. Then we have $|\mathcal{S}_0| = |\mathcal{S}_1| = |\mathbb{B}^\omega| = 2^{\aleph_0} =: \kappa$. Let $(f_\alpha)_{\alpha < \kappa}$ and $(g_\alpha)_{\alpha < \kappa}$ be an enumeration of $\mathcal{S_0}$ and $\mathcal{S}_1$ respectively. For $f \in \mathcal{S}_0, g \in \mathcal{S}_1$, we write $\langle f, g \rangle \in \mathbb{B}^\omega$ for the unique play according to $f$ and $g$.
	
	Our goal is to construct a family of sets $(L_\alpha, M_\alpha)_{\alpha < \kappa} \in \mathbb{B}^\omega \times \mathbb{B}^\omega$ such that for all $\alpha < \kappa$:
	\begin{enumerate}
		\item for all $\beta < \alpha$: $L_\beta \subseteq L_\alpha$ and $M_\beta \subseteq M_\alpha$
		\item $L_\alpha \cap M_\alpha = \emptyset$
		\item $|L_\alpha| = |M_\alpha| = \alpha$
		\item for all $\beta < \alpha$ there is an $f \in \mathcal{S}_0$ such that $\langle f, g_\beta \rangle \in L_\alpha$
		\item for all $\beta < \alpha$ there is an $g \in \mathcal{S}_1$ such that $\langle f_\beta, g \rangle \in M_\alpha$
	\end{enumerate}
	
	If that is done, set $L := \bigcup\limits_{\alpha < \kappa} L_\alpha$. We claim that $\Gamma(L)$ is not determined. Assume player 0 has a winning strategy $f^* \in \mathcal{S}_0$, so $f^* = f_\alpha$ for some $\alpha < \kappa$. By (5), there must be a $g \in \mathcal{S}_1$ such that $\langle f^*, g \rangle \in M_{\alpha+1}$. Because of (1) and (2), $\langle f^*, g \rangle \notin L$, so this play is won by player 1. Hence, $f^*$ cannot be a winning strategy. 
	
	\textbf{Claim}: Sets $L_\alpha, M_\alpha$ as described above exist.
	
	For $\alpha = 0$, set $L_\alpha = M_\alpha = \emptyset$. Otherwise, let $0 < \alpha < \kappa$ be arbitrary. We find plays $\pi_L, \pi_M \in \mathbb{B}^\omega$ such that $L_\alpha = \bigcup\limits_{\beta < \alpha} L_\beta \cup \{\pi_L\}$ and $M_\alpha = \bigcup\limits_{\beta < \alpha} M_\beta \cup \{\pi_M\}$ satisfy the conditions.
	
	Let $P = \bigcup\limits_{\beta < \alpha} (L_\beta \cup M_\beta)$. Let $g \in \mathcal{S}_1$ such that $\pi_L := \langle f_\alpha, g \rangle \notin P$. To see that this is possible, note that $|P| = 2 \alpha < \kappa$ because of (3). Analogously, find an $f \in \mathcal{S}_0$ such that $\pi_M := \langle f, g_\alpha \rangle \notin P \cup \{\pi_L\}$. 
\end{proof}


\end{document}
















