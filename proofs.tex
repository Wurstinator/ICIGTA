\documentclass{article}

\usepackage{amsmath, amssymb, amsthm}
\usepackage[utf8]{inputenc}
\usepackage{geometry}
\newtheorem{theorem}{Theorem}
\newtheorem{definition}{Definition}
\usepackage{xr}
\usepackage{bm}

\begin{document}

\section{Infinite Computations}
\begin{theorem}
\label{nba_complement}
	NBA-recognizable languages are closed under complement.
\end{theorem}
\begin{proof}
	%TODO F4
\end{proof}


\vspace{1cm}
\begin{theorem}
\label{ltl_to_nba}

\end{theorem}
\begin{proof}
	%TODO F6-8
\end{proof}


\vspace{1cm}
\begin{theorem}[Landweber]
\label{landweber}
	Let $\mathcal{A} = (Q, \Sigma, q_0, \delta, \mathcal{F})$ be a DMA. 
	\begin{enumerate}
		\item $L(\mathcal{A})$ is DBA-recognizable iff $\mathcal{F}$ is closed under super loops.
		\item $L(\mathcal{A})$ is E-recognizable iff $\mathcal{F}$ is closed under reachable loops.
	\end{enumerate}
\end{theorem}
\begin{proof}
	%TODO F16-18
\end{proof}


\vspace{1cm}
\begin{theorem}
\label{mso_nba}
	NBAs, S1S, S1S\textsubscript{0}, and $\exists$S1S have the same expressive power.
\end{theorem}
\begin{proof}
	\begin{description}
	\item[$\bm{\text{NBA} \Rightarrow \exists \text{S1S}}$] Let $\mathcal{A} = (Q, \mathbb{B}^n, 1, \Delta, F)$ be the NBA with $Q = \{1, \dots, m\}$. We encode a run of $\mathcal{A}$ on a word with a formula. 
	
	$$\varphi_\mathcal{A}(X_1, \dots, X_n) = \exists Y_1 \dots \exists Y_m \text{Part}(\overline{Y}) \land Y_1(0) \land \text{Trans}(\overline{X}, \overline{Y}) \land \text{Fin}(\overline{Y}) $$
	$$ \text{Part}(Y_1, \dots, Y_m) = \forall x \left( \bigvee\limits_{i=1}^m Y_i x \land \bigwedge\limits_{i=1}^m \bigwedge\limits_{j \neq i} \neg Y_i x \lor \neg Y_j x \right) $$
	$$ \text{Trans}(\overline{X}, \overline{Y}) = \forall x \bigvee\limits_{\tau \in \Delta} \psi_\tau(x, \overline{X}, \overline{Y}) $$
	$$ \psi_{(p, a, q)}(x, X_1, \dots, X_n, Y_1, \dots, Y_m) = Y_p x \land Y q (x+1) \land X_a x $$
	$$ \text{Fin}(Y_1, \dots, Y_m) = \forall x \exists y \left( x < y \land \bigvee\limits_{q \in F} Y_q y \right) $$
	
	\item[$\bm{\exists \text{S1S} \Rightarrow \text{S1S}}$] trivial
	
	\item[$\bm{\text{S1S} \Rightarrow \text{S1S}_0}$] Let $\varphi \in \text{S1S}$. We define an equivalent $\varphi' \in \text{S1S}_0$ inductively.
	
	\begin{itemize}
		\item Boolean operators stay the same.
		\item $\exists x \psi(x) \mapsto \exists X_x (\text{Sing}(X_x) \land \psi'(X_x))$
		\item $Xx \mapsto X_x \subseteq X$
		\item $x+1 = y \mapsto \text{Succ}(X_x, X_y)$
	\end{itemize}
	
	\item[$\bm{\text{S1S}_0 \Rightarrow \text{NBA}}$] $\exists$, $\lor$, $\neg$ correspond to projection, union, and complement of automata respectively. Therefore it suffices to give NBAs for the three base formulas.
	
	\begin{itemize}
		\item $X_1 \subseteq X_2$: $Q = F = \{q_0\}$, $\Delta = \{(q_0, a, q_0) \mid a \neq (1 \; 0)\}$.
		\item $\text{Sing}(X)$: $Q = \{q_0, q_f\}$, $F = \{q_f\}$, $\Delta = \{(q_0, 0, q_0), (q_f, 0, q_f), (q_0, 1, q_f)\}$.
		\item $\text{Succ}(X_1, X_2)$: $Q = \{q_0, q_1, q_2\}$, $F = \{q_2\}$, \\
		$\Delta = \{ (q_0, (0 \; 0), q_0), (q_2, (0 \; 0), q_2), (q_0, (1 \; 0), q_1), (q_1, (0 \; 1), q_2)\}$.
	\end{itemize}
	
	The complexity of this construction is $2 \uparrow \uparrow |\varphi|$.
	\end{description}
\end{proof}


\vspace{1cm}
\begin{theorem}[McNaughton]
\label{nba_dma}
	A language is NBA-recognizable iff it is DMA-recognizable.
\end{theorem}
\begin{proof}
	$\bm{\Leftarrow}$ NBA with $L(\mathcal{A}) = \bigcup\limits_{F \in \mathcal{F}} \left( \bigcap\limits_{q \in F} L(\mathcal{A}_q) \cap \bigcap\limits_{q \notin F} \overline{L(\mathcal{A}_q)} \right)$ where $\mathcal{A}_q$ is $\mathcal{A}$ starting in $q$.
	
    $\bm{\Rightarrow}$ %TODO F12-13 notes
\end{proof}


\vspace{1cm}
\begin{theorem}
\label{muller_to_parity}
	For each Muller automaton, one can construct an equivalent parity automaton. The construction preserves determinism.
\end{theorem}
\begin{proof}
	%TODO F14-15 notes?
\end{proof}


\vspace{1cm}
\begin{theorem}
\label{aba_to_nba}
	For each ABA, one can construct an equivalent NBA with states at most $3^{|Q|}$.
\end{theorem}
\begin{proof}
	Let $\mathcal{A} = (Q, \Sigma, q_0, \delta, F)$ be an ABA. We define $\mathcal{A}' = (\{0,1,2\}^Q, \Sigma, f_0, \Delta, \{0,2\}^Q)$ with $f_0(q) = \begin{cases} 1 & \text{if } q = q_0 \\ 0 & \text{else} \end{cases}$ and $\Delta$ as described below. 
	
	A state is a function $f : Q \rightarrow \{0, 1, 2\}$. Consider a run-tree of $\mathcal{A}$ on some word and all states that are ``active'' on one level in the tree. $f(q)$ is 0 for all states which are not active and 1 or 2 for all others. A state is assigned to 1 by default, which just means that it is active in some path. It is assigned 2 if ``recently'' on all paths it is active on, a final state in $F$ was seen. Recently means that it occurred since the last time that an entire level in the run tree was made up of final states.
	
	Formally, we have $(f, a, g) \in \Delta$ for all $g : Q \rightarrow \{0, 1, 2\}$ which satisfy the following:
	\begin{itemize}
		\item The active states need to be passed on, i.e. for all $q \in Q$: if $f(q) \in \{1,2\}$ then there must be an $X_q \subseteq Q$ with $X_q \models \delta(q, a)$ such that $g(X_q) \subseteq \{1,2\}$.
		\item A state is assigned 2 if it is final, i.e. $g(q) = 2$ if $q \in F$.
		\item Otherwise, a 1 overwrites a 2 for succession, i.e. for all $p \in P$, if there is a $q$ with $f(q) = 1$ and $p \in X_q$, then also $g(p) = 1$.
		\item If all states are marked with a 2, $\mathcal{A}'$ reached a final state. We reset the values to $g(p) = 2$ iff $p \in F$.
	\end{itemize}
\end{proof}



\newpage
\section{Tree Automata}
\begin{theorem}
\label{twa_to_nta}
	Let $\mathcal{A} = (Q, \Sigma, q_0, \Delta, F)$ be a TWA. There is an NTA $\mathcal{A}'$ of size exponential in $|Q|$ that recognizes $T(\mathcal{A})$.
\end{theorem}
\begin{proof}
	Let $\sim_{T(\mathcal{A})}$ be the usual equivalence relation, i.e. $t_1 \sim_{T(\mathcal{A})} t_2$ iff $\forall s \in S_\Sigma: s \cdot t_1 \in T(\mathcal{A}) \leftrightarrow s \cdot t_2 \in T(\mathcal{A})$. We define a relation $\sim \subseteq T_\Sigma \times T_\Sigma$ such that $\text{index}(\sim_{T(\mathcal{A})}) \leq \text{index}(\sim) \leq 2^{|Q|^2 \cdot m + 1}$, where $m$ is the maximal rank in $\Sigma$.
	
	Let $t_0 \in T_\Sigma$ and $a_m \in \Sigma_m$ be arbitrary. For every $t \in T_\Sigma$ and $1 \leq i \leq m$, we define $t^{(i)} = a_m(t_0, t_0, \dots, t, \dots, t_0)$, meaning the $i$-th subtree below the root is $t$. Further, we define a relation $B_t^i \subseteq Q \times Q$ with $(p, q) \in B_t^i$ iff there is a run segment $\rho$ of $\mathcal{A}$ on $t^{(i)}$, such that the run begins at the root of $t$, never leaves that subtree until the end. Meaning, $\rho = (p, i) (q_1, i u_1) \dots (q_n, i u_n) (q, \varepsilon)$.
	
	Finally, let $t_1 \sim t_2$ iff $t_1 \in T(\mathcal{A}) \leftrightarrow t_2 \in T(\mathcal{A})$ and $\forall i: B^i_{t_1} = B^i_{t_2}$.
	
	Idea: $(p, q) \in B^i_t$ if $\mathcal{A}$ can enter $t$ as $i$-th child with state $p$ and after some while leaves it again with state $q$.
	
	\paragraph{Claim}: Let $t_1 \sim t_2$. Then $t_1 \sim_{T(\mathcal{A})} t_2$.
	
	Let $s \in S_\Sigma$. Due to the symmetric definition of $\sim$, it suffices to show that $t_1 \in T(\mathcal{A})$ implies $t_2 \in T(\mathcal{A})$, so let $t_1 \in T(\mathcal{A})$. If $s = \circ$, then $s \cdot t_1 = t_1 \in T(\mathcal{A})$. By definition of $\sim$, this implies $s \cdot t_2 = t_2 \in T(\mathcal{A})$.
	
	Otherwise $s \neq \circ$. Let $\rho_1 \rho_2 \rho_3$ be an accepting run of $\mathcal{A}$ on $s \cdot t_1$ such that $\rho_1$ only stays outside of $t_1$ and $\rho_2$ only stays inside of $t_1$. Since $B_{t_1}^i = B_{t_2}^i$, there is a run segment of $\mathcal{A}$ on $t_2$ which enters and exits the tree with the same states as $\rho_2$ does, meaning it can replace $\rho_2$ in the accepting run. Repeating this procedure gives an accepting run of $\mathcal{A}$ on $s \cdot t_2$, so $t_2 \in T(\mathcal{A})$.
	
	\paragraph{Notes} on the construction: each state in the NTA corresponds to a list of $Q$-states that $\mathcal{A}$ had when visiting this node, together with the direction from which it was coming. That can be used to check the correctness of a run.
\end{proof}

\vspace{1cm}
\begin{theorem}
\label{tre_fromto_nta}
	A language of finite trees $T \subseteq T_\Sigma$ can be recognized by an NTA iff it can be described by a regular tree expression.
\end{theorem}
\begin{proof}
	$\bm{\Rightarrow}$ Let $\mathcal{A} = (Q, \Sigma, \Delta, F)$ be an NTA for $T$. For $R, I \subseteq Q, q \in Q$, we define $T(R, I, q) \subseteq T_{\Sigma \cup C_R}$ (where $C_R = \{c_p \mid p \in R\}$) as the set of all trees on which $\mathcal{A}$ has a run that only uses states in $I$ and ends in $q$. We can inductively define regular expressions for $T(R, I, q)$. Then $T(\mathcal{A}) = \bigcup\limits_{q \in F} T(\emptyset, Q, q)$.
	
	For all $R \subseteq Q, q \in Q$, $T(R, \emptyset, q)$ contains only trees of height 0 or 1, so it is a finite union of singular languages.
	
	$$T(R, I \cup \{i\}, q) = T(R, I, q) + T(R \cup \{i\}, I, q) \cdot^{c_i} (T(R \cup \{i\}, I, i))^{*_{c_i}} \cdot^{c_i} T(R, I, i)$$ \\
	
	$\bm{\Leftarrow}$ Show by induction that a regular expression $r$ can be transformed to a NTA $\mathcal{A}_r$.
	\begin{itemize}
		\item If $r = t \in T_{\Sigma \cup C}$, then there is an automaton $\mathcal{A}_t$ with $T(\mathcal{A}_t) = \{t\}$.
		\item If $r = s + t$, then $\mathcal{A}_r$ is the union automaton of $\mathcal{A}_s$ and $\mathcal{A_t}$.
		\item If $r = s \cdot^c t$, then $\mathcal{A}_r = (Q_s \cup Q_t, \Sigma \cup C, \Delta_r, F_s)$, where $\Delta_r$ simulates $\mathcal{A}_s$ and $\mathcal{A}_t$ but the final transitions in $\mathcal{A}_t$ are replaced by initial transitions to $\mathcal{A}_s$.
		\item If $r = s^{+_c}$, let $\mathcal{A}_r = (Q_s, \Sigma \cup C, \Delta_r, F_s)$, where $\Delta_r$ simulates $\mathcal{A}_s$ and allows ``restarts'' when a final state could be reached.
	\end{itemize}
\end{proof}

\vspace{1cm}
\begin{theorem}
\label{edtd_complto_nuta}
	Let $D$ be an EDTD. We can construct a NUTA $\mathcal{A}$ with $T(D)^\complement = T(\mathcal{A})$ in polynomial time in $|D|$.
\end{theorem}
\begin{proof}
	Let $D = (\Sigma', P, S^{(1)})$. For $a^{(i)} \in \Sigma'$, let $a^{(i)} \rightarrow r_{a,i}$ be the according rule in $P$, where $r_{a,i}$ is a regular expression that can be transformed to a DFA in polynomial time. If $b \in \Sigma$ occurs in $r_{a,i}$ then it must have a unique type $j$ (since $D$ is single-typed). We call $b^{(j)}$ the $b$-successor of $a^{(i)}$. If $b$ does not occur in $r_{a,i}$, we say that the $b$-successor is $b^\perp$. Furthermore, we assume $r_{a,\perp} = \varepsilon$ for all $a$.
	
	We define a typing function $f : \text{dom}_t \rightarrow \{\perp, 1, \dots, k\}$. We assign $f(\varepsilon) = \begin{cases} 1 & \text{if } \text{val}_t(\varepsilon) = S \\ \perp & \text{else} \end{cases}$. For the other nodes, let $u$ be a node with parent $v$. We call $\text{val}_t(v) = a$ and $\text{val}_t(u) = b$. Then there is a unique $i \in \{\perp, 1, \dots, k\}$ such that $b^{(i)}$ is the $b$-successor of $a^{f(v)}$. We set $f(u) := i$.
	
	\textbf{Claim}: $t \in T(D)$ iff $\forall v \in \text{dom}_t: f(v) \neq \perp$ and $a_1^{f(v1)} \dots a_m^{f(vm)} \in L(r_{a,f(v)})$, where $a = \text{val}_t(v)$ and $a_j = \text{val}_t(vj)$. (without proof)
	
	Using this claim, we can provide an automaton $\mathcal{A} = (Q, \Sigma, \Delta, F)$ that accepts a tree iff it violates the constraint.
	
	\begin{itemize}
		\item $Q = \Sigma \times \{\perp, 1, \dots, k\} \times \{0, 1\}$, where the last component denotes whether a violation was found.
		\item $\Delta = \{ (L_{a,i,x}, a, (a, i, x)) \mid a \in \Sigma, i \in \{\perp, 1, \dots, k\}, x \in \{0,1\} \}$ \\
			Let $\alpha = (a_1, i_1, x_1) \dots (a_m, i_m, x_m) \in Q^m$ and $w := a_1^{(i_1)} \dots a_m^{(i_m)} \in (\Sigma')^m$.
			\begin{itemize}
				\item $\text{succ} :\Leftrightarrow$ for all $1 \leq j \leq m$, $a_j^{(i_j)}$ is the $a_j$-successor of $a^{(i)}$.
				\item $\text{sat}_0 :\Leftrightarrow$ $w \in L(r_{a,i})$ and for all $1 \leq j \leq m$, $x_j = 0$ and $i_j \neq \perp$.
				\item $\text{sat}_1 :\Leftrightarrow$ $w \notin L(r_{a,i})$ or there is a $1 \leq j \leq m$ such that $x_j = 1$ or $i_j = \perp$.
			\end{itemize}
			Then $\alpha \in L_{a,i,x}$ iff $\text{succ}$ and $\text{sat}_x$ hold.
		\item $F = \{ (S, 1, 1) \} \cup \{ (a, i, x) \mid a \neq S \}$, meaning either a violation was found or the starting symbol was not $S$.
\end{itemize}
\end{proof}

\vspace{1cm}
\begin{theorem}
\label{dtwa_complement}
	The class of DTWA-recognizable languages is closed under complement.
\end{theorem}
\begin{proof}
	Let $\mathcal{A} = (Q, \Sigma, q_0, \delta, \{q_f\})$ be a DTWA that only moves to $q_f$ at the root. The \textbf{backwards configuration graph} $\text{BCG}(\mathcal{A}, t)$ is defined as a tree over $Q \times \text{dom}_t$ with root $(q_f, \varepsilon)$. For a node $(q, u)$, the children are all $(p, v)$ such that $(p, v) \rightarrow_\mathcal{A} (q, u)$. We define $\overline{\mathcal{A}}$ in a way that it performs DFS on the BCG of the input tree and accepts iff the node $(q_0, \varepsilon)$ is found.
	
	For that, let $\prec \subseteq (Q \times \text{Dir})^2$ be an arbitrary linear order on $Q \times \text{Dir}$. We set $\overline{\mathcal{A}} = (\overline{Q}, \Sigma, q_0, \overline{\delta}, q_f)$ with $\overline{Q} = \{q_0, q_f\} \cup \{ \langle p, (q, d) \rangle \mid p, q \in Q, d \in \text{Dir} \}$. The behavior of $\overline{\delta}$ is described below. Let $\langle q, (q', d) \rangle$ be a state.
	
	\paragraph{Case 1}: In the ordering $\prec$, $(\hat{q}, \hat{d})$ is the next largest element after $(q', d)$. (for $q_0$ we also consider this case with the $\prec$-minimal pair.)
	
	%TODO
	
	\paragraph{Case 2}: $(q', d)$ %TODO
\end{proof}

\vspace{1cm}
\begin{theorem}
\label{fcns_correct}
	Let $T \subseteq T_\Sigma$. $T$ is regular iff $\text{fcns}(T)$ is regular.
\end{theorem}
\begin{proof}
	$\boldsymbol{\Rightarrow}$ Let $\mathcal{A} = (Q, \Sigma, \Delta, F)$ be a NUTA with $T(\mathcal{A}) = T$. Wlog we assume that $\mathcal{A}$ is normalized. For every transition $\tau = (L_{a,q}, a, q) \in \Delta$, let $\mathcal{B}_{a,q} = (P_{a,q}, Q, p^0_{a,q}, \Delta_{a,q}, F_{a,q})$ be a NFA with $L(\mathcal{B}_{a,q}) = L_{a,q}$. We define $\mathcal{A}_\text{fcns} = (Q_\text{fcns}, \Gamma, \Delta_\text{fcns}, F_\text{fcns})$ so that $T(\mathcal{A}_\text{fcns}) = \text{fcns}(T)$.
	
	\begin{itemize}
		\item $Q_\text{fcns} = \{q_f, q_\#\} \cup \bigcup\limits_{a \in \Sigma, q \in Q} P_{a, q}$
		\item $F_\text{fcns} = \{q_f\}$
		\item $\Delta_\text{fcns}$:
		\begin{itemize}
			\item $(\#, q_\#)$
			\item For all $p \in \bigcup\limits_{a \in \Sigma, q \in Q} F_{a, q}$: $(\#, p)$
			\item For all $a \in \Sigma, q \in F$: $(p^0_{a,q}, q_\#, a, q_f)$
			\item For all $a \in \Sigma, p \in P, p' \in P, q \in Q$ with $(p, q', p') \in \bigcup\limits_{b \in \Sigma, q' \in Q} \Delta_{b, q'}$: $(p^0_{a,q}, p', b, p)$
		\end{itemize}
	\end{itemize}
	
	Via induction on $t_1 \dots t_n$, one can show that $p \in \Delta^*_\text{fcns}(\text{fcns}(t_1 \dots t_n))$ iff there are $q_1, \dots q_n \in Q$ such that $\forall i: q_i \in \Delta^*(t_i)$ and $q_1 \dots q_n \in L(\mathcal{B}_{a,q} \text{ init } p)$.
	
	$\boldsymbol{\Leftarrow}$ Let $\mathcal{A} = (Q, \Sigma, \Delta, F)$ be an NTA with $T(\mathcal{A}) = \text{fcns}(T)$. We define %no proof?
\end{proof}

\end{document}
















