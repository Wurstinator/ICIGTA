\documentclass{article}

\usepackage{amsmath, amssymb, amsthm}
\usepackage[utf8]{inputenc}
\usepackage{geometry}

\newtheorem{theorem}{Theorem}[section]
\newtheorem{definition}{Definition}

\begin{document}

\section{Infinite Words}

\begin{theorem}
	Every non-empty $\omega$-regular language contains an ultimately periodic word.
\end{theorem}
%TODO F4


\begin{theorem}
	For a Kripke structure $\mathcal{K}$ with initial state $s$ and $\varphi \in \text{LTL}$, the model checking problem $L(\mathcal{K}, s) \subseteq L(\varphi)?$ is PSPACE-complete.
\end{theorem}
\begin{proof}
	\textbf{PSPACE} Compute the intersection automaton for $L(\mathcal{K}, s) \cap L(\neg \varphi)$ and test it for emptiness.
	
	\textbf{PSPACE-hard} Encode a poly.-length Turing tape as a Kripke structure and its correct behavior in LTL.
\end{proof}

\begin{theorem}[Büchi]
	The MSO theory of $(\mathbb{N}, +1, <, 0)$ is decidable.
\end{theorem}
\begin{proof}
	Corresponds to S1S formula. Can be checked with NBA emptiness test.
\end{proof}

\begin{theorem}
	The FO theory of $(\mathbb{R}, +, <, 0)$ is decidable.
\end{theorem}
\begin{proof}
	Encode real numbers $x$ by triples of sets $(X_s, X_i, X_f)$ with the number's sign ($X_s = \emptyset$ or $\{0\}$), the positive decimal digits in binary encoding, and the positive fractional digits in binary encoding. Then an FO sentence can be transformed to an equi-satisfiable MSO sentence over $(\mathbb{N}, +1, <, 0)$.
\end{proof}

\begin{theorem}
	Subset-construction does not suffice to determinize NBAs.
\end{theorem}
%TODO F12

\begin{theorem}
	For every $n$, there is $L_n \subseteq \Sigma^\omega$ s.t. there is an NBA that recognizes $L_n$ with $n+2$ states, but every det. Rabin automaton that recognizes $L_n$ has at least $n!$ states.
\end{theorem}
%TODO F14

\begin{theorem}
	There is a DBA-recog. language which does not have a unique minimal DBA. \\
	DBAs minimized with the DFA minimization algorithm can be arbitrarily bad compared to a minimal DBA.
\end{theorem}
%TODO F18

\begin{theorem}
	Weak DBAs can be minimized uniquely in polynomial time.
\end{theorem}
%TODO F19

\begin{theorem}
	Given an ABA $\mathcal{A}$, the dual $\tilde{\mathcal{A}}$ is an alternating co-Büchi automaton which accepts $\overline{L(\mathcal{A})}$, with $\tilde{F} = Q \setminus F$ and $\tilde{\delta}$ exchanging true/false and $\land$/$\lor$.
\end{theorem}
%TODO F22

\subsection{Simulation Game}
\begin{definition}
	Let $\mathcal{A} = (Q, \Sigma, q_0, \Delta, F)$ be an NBA. We define the delayed simulation game $\mathcal{G}_\mathcal{A}(G_\mathcal{A}, \text{Win})$ as follows
	\begin{itemize}
		\item $G_\mathcal{A} = (V_0, V_1, E, c)$ 
		\item $V_0 = \Sigma \times Q \times Q$
		\item $V_1 = Q \times Q$
		\item Player 0 moves from $(a, p, q)$ to a $(p, p')$ with $(q, a, p') \in \Delta$
		\item Player 1 moves from $(q, q')$ to a $(a, p, q')$ with $(q, a, p) \in \Delta$
		\item $c : V \rightarrow \{-1, 0, 1\}$ with $c(v) = \begin{cases} -1 & \text{if } v \in F \times (Q \setminus F) \\ 1 & \text{if } v \in Q \times F \\ 0 & \text{otherwise} \end{cases}$
		\item $\alpha \in \text{Win}$ iff after every -1 in $\alpha$, there is a 1 later on.
	\end{itemize}
	
	Write $q \preceq_\text{de} q'$ if player 0 has a winning strategy from $(q, q')$.
\end{definition}

Idea: Player 1 chooses symbols in $\Sigma$ and transitions on the second state. Player 0 has to answer with transition on the first state which lead to a run of the same acceptance.

\begin{theorem}
	If $q \preceq_\text{de} q'$ and $q' \preceq_\text{de} q$, then $q$ and $q'$ can be merged in $\mathcal{A}$ without changing the language of the automaton.
\end{theorem}
% no proof

\begin{theorem}
	The delayed simulation game can be reduced to a Büchi game in linear time.
\end{theorem}
 

\section{Finite Trees}
\begin{theorem}[Pumping Principle]
	Let $T \subseteq T_\Sigma$ be a regular ranked tree language. There is a $n \in \mathbb{N}$ such that for all trees $t \in T$, all $m > n$, and all paths $\pi_1 \dots \pi_m$, there are $1 \leq i < j \leq m$ such that for all $k \in \mathbb{N}$:
	$$t[\circ / u] \cdot (t[\circ/v]|_u)^k \cdot t|_v \in T$$
	where $u = \pi_1 \dots \pi_i$ and $v = \pi_1 \dots \pi_j$.
\end{theorem}
% no proof

\begin{definition}
	Let $T \subseteq T_\Sigma$. The \textbf{Myhill-Nerode equivalence} is $\sim_T \subseteq T_\Sigma \times T_\Sigma$ with 
	$$t_1 \sim_T t_2 \Leftrightarrow \forall s \in S_\Sigma: s \cdot t_1 \in T \leftrightarrow s \cdot t_2 \in T$$
	
	The index of $T$ is $\text{Index}(\sim_T) := |T/\sim_T|$.
\end{definition}

\begin{definition}
	Let $T \subseteq T_\Sigma$. We define the canonical DTA $\mathcal{A}_T = (Q_T, \Sigma, \delta_T, F_T)$ as
	\begin{itemize}
		\item $Q_T = \{ [t]_{\sim_T} \mid t \in T_\Sigma \}$.
		\item For all $a \in \Sigma_i$: $\delta_T([t_1]_{\sim_T}, \dots, [t_i]_{\sim_T}, a) = [a(t_1, \dots, t_i)]_{\sim_T}$.
		\item $F_T = \{ [t]_{\sim_T} \mid t \in T \}$.
	\end{itemize}
\end{definition}

\begin{theorem}
	Let $T \subseteq T_\Sigma$. $T$ is regular iff $\text{Index}(\sim_T)$ is finite. If $T$ is regular, $\mathcal{A}_T$ is the minimal DTA.
\end{theorem}
\begin{proof}
	via induction on $t$, prove $\delta_T^*(t) = [t]$
\end{proof}

\begin{theorem}
	The emptiness problem for NTAs can be reduced to \textsf{HORN-SAT} in linear time.
\end{theorem}
\begin{proof}
	Let $\mathcal{A} = (Q, \Sigma, \Delta, F)$ be an NTA. For every $\tau = (q_1, \dots, q_i, a, p) \in \Delta$, let \linebreak $\psi_\tau = (X_{q_1} \land \dots \land X_{q_i} \rightarrow X_q)$. Then we define $\varphi = \bigwedge\limits_{\tau \in \Delta} \psi_\tau \land \bigwedge\limits_{q \in F} X_q \rightarrow 0$. $\varphi$ is satisfiable iff $L(\mathcal{A}) \neq \emptyset$.
\end{proof}

\subsection{BTTs}
\begin{theorem}
	The equivalence problem for BTTs is undecidable.
\end{theorem}
%TODO F13

\begin{theorem}
	The emptiness problem for BTTs is decidable in polynomial time.
\end{theorem}
%TODO F13

\begin{theorem}
	The type-checking problem (given regular $T, T'$, is $\mathcal{A}(T) \subseteq T'$?) is decidable.
\end{theorem}
%TODO F13

\begin{theorem}
	If $T$ is regular, then $\mathcal{A}^{-1}(T)$ is regular. \\
	If $\mathcal{A}$ is linear, then $\mathcal{A}(T_\Sigma)$ is regular.
\end{theorem}
%no proof

\begin{theorem}
	There are BTT-definable relations $R_1, R_2$ such that $R_1 \circ R_2$ is not BTT-definable.
\end{theorem}
%TODO F13

\begin{theorem}
	If $\mathcal{A}_1$ is linear \textbf{or} $\mathcal{A}_2$ is deterministic Then $R(\mathcal{A}_1) \circ R(\mathcal{A}_2)$ is BTT-definable.
\end{theorem}
%TODO F13


\section{Infinite Trees}
\begin{theorem}[BTA Pumping]
	For $t \in T_\Sigma, x \in \{0,1\}^*, y \in \{0,1\}^+$, let 
	$$t^*_{[x,y]} : \{0,1\}^* \rightarrow \Sigma, z \mapsto \begin{cases} t(z) & \text{if } xy \not\sqsubseteq z \\ xz' & \text{if } \exists n>0: z = xy^nz' \text{ with } y \not\sqsubseteq z' \end{cases}.$$
	Let $\mathcal{A}$ be a BTA, $t \in T(\mathcal{A})$, $\rho$ an accepting run of $\mathcal{A}$ on $t$, and $x, y, y' \in \{0,1\}^*$ s.t. $\rho(x) = \rho(xy)$, $y' \sqsubset y$, and $\rho(xy') \in F$. Then $t^*_{[x,y]} \in T(\mathcal{A})$.
\end{theorem}
%TODO F14

\begin{theorem}
	Every non-empty regular tree language contains a regular tree.
\end{theorem}
%TODO F18

\begin{theorem}[Rabin's Tree Theorem]
	The MSO theory of $\underline{T_2}$ is decidable for formulas $\varphi(X_1, \dots, X_n)$ and a model $X_1, \dots X_n \subseteq \{0,1\}^*$ is computable.
\end{theorem}
\begin{proof}
	Transform $\varphi$ into an equivalent PTA. A model can be found by solving the emptiness game.
\end{proof}


\end{document}
















