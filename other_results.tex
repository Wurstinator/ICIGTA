\documentclass{article}

\usepackage{amsmath, amssymb, amsthm}
\usepackage[utf8]{inputenc}
\usepackage{geometry}
\usepackage{bm}

\newtheorem{theorem}{Theorem}[section]
\newtheorem{definition}{Definition}

\begin{document}

\section{Infinite Words}

\vspace{.5cm}
\begin{theorem}
	Every non-empty $\omega$-regular language contains an ultimately periodic word.
\end{theorem}
\begin{proof}
	Let $U$ be $\omega$-regular, so there is a regular expression $r = \bigcup_{i=1}^n U_i \cdot V_i^\omega$ for it. Since $U \neq \emptyset$, $n$ must be larger than 0. Let $u \in U_1$ and $v \in V_1$ be arbitrary. Then $u v^\omega \in U$.
\end{proof}


\vspace{.5cm}
\begin{theorem}
	For a Kripke structure $\mathcal{K}$ with initial state $s$ and $\varphi \in \text{LTL}$, the model checking problem $L(\mathcal{K}, s) \subseteq L(\varphi)?$ is PSPACE-complete.
\end{theorem}
\begin{proof}
	\textbf{PSPACE} Compute the intersection automaton for $L(\mathcal{K}, s) \cap L(\neg \varphi)$ and test it for emptiness.
	
	\textbf{PSPACE-hard} Encode a poly.-length Turing tape as a Kripke structure and its correct behavior in LTL.
\end{proof}

\vspace{.5cm}
\begin{theorem}[Büchi]
	The MSO theory of $(\mathbb{N}, +1, <, 0)$ is decidable.
\end{theorem}
\begin{proof}
	Corresponds to S1S formula. Can be checked with NBA emptiness test.
\end{proof}

\vspace{.5cm}
\begin{theorem}
	The FO theory of $(\mathbb{R}, +, <, 0)$ is decidable.
\end{theorem}
\begin{proof}
	Encode real numbers $x$ by triples of sets $(X_s, X_i, X_f)$ with the number's sign ($X_s = \emptyset$ or $\{0\}$), the positive decimal digits in binary encoding, and the positive fractional digits in binary encoding. Then an FO sentence can be transformed to an equi-satisfiable MSO sentence over $(\mathbb{N}, +1, <, 0)$.
\end{proof}

\vspace{.5cm}
\begin{theorem}
	Subset-construction does not suffice to determinize NBAs.
\end{theorem}
\begin{proof}
	Example: Let $\mathcal{A} = (Q, \{a\}, q_0, \Delta, F)$ be an NBA with $Q = \{q_0, q_1\}$, $F = \{q_1\}$, and $\Delta = \{(q_0, a, q_0), (q_0, a, q_1)\}$. Then $L(\mathcal{A}) = \emptyset$.
	
	The subset construction yields $\mathcal{A}' = \{ \{ \{q_0\}, \{q_0, q_1\} \}, \{a\}, \{q_0\}, \Delta, \{ P \subseteq Q \mid P \cap F \neq \emptyset\})$ with $\Delta = \{ (\{q_0\}, a \{q_0, q_1\}), (\{ q_0, q_1 \}, a, \{ q_0, q_1\} ) \}$. Therefore, the word $a^\omega$ has an accepting run $\{q_0\} \{q_0, q_1\}^\omega$.
\end{proof}


\vspace{.5cm}
\begin{theorem}
\label{union_lemma}
	Let $\mathcal{A}$ be a Rabin automaton and let $\rho_1, \rho_2, \rho$ be runs of $\mathcal{A}$. If $\text{Inf}(\rho_1) \cup \text{Inf}(\rho_2) = \text{Inf}(\rho)$ and $\rho_1$ and $\rho_2$ are both non-accepting, then $\rho$ is also non-accepting.
\end{theorem}
\begin{proof}
	Let $\Omega = \{ (E_i, F_i) \mid 1 \leq i \leq m \}$. Let $I_1 = \text{Inf}(\rho_1)$, $I_2 = \text{Inf}(\rho_2)$, and $I = \text{Inf}(\rho)$. Assume $\rho$ is accepting, so there is an $1 \leq i \leq m$ s.t. $I \cap E_i = \emptyset$ and $I \cap F_i \neq \emptyset$. That means $I \cap E_i = (I_1 \cup I_2) \cap E_i = \emptyset$, so $I_1 \cap E_i = I_2 \cap E_i = \emptyset$. Since $\rho_1$ and $\rho_2$ are non-accepting, that means $I_1 \cap F_i = I_2 \cap F_i = \emptyset$ and therefore $I \cap F_i = \emptyset$, which is a contradiction.
\end{proof}


\vspace{.5cm}
\begin{theorem}
	For every $n$, there is $L_n \subseteq \Sigma^\omega$ s.t. there is an NBA that recognizes $L_n$ with $n+2$ states, but every det. Rabin automaton that recognizes $L_n$ has at least $n!$ states.
\end{theorem}
\begin{proof}
	We define the NBA $\mathcal{A}_n = (Q, \Sigma_n, q_0, \Delta, F)$ with
	\begin{itemize}
		\item $\Sigma_n = \{\#, 1, \dots, n\}$
		\item $Q = \{q_0, q_f, q_1, \dots, q_n\}$
		\item $F = \{q_f\}$
		\item $\Delta$:
		\begin{itemize}
			\item For all $1 \leq i \leq n$, for all $a \in \Sigma_n$: $q_i \overset{a}{\rightarrow} q_i$ and $q_0 \overset{a}{\rightarrow} q_i$
			\item For all $1 \leq i \leq n$: $q_i \overset{i}{\rightarrow} q_f$ and $q_f \overset{i}{\rightarrow} q_i$
		\end{itemize}
	\end{itemize} % lec 14 
	
	Then $\alpha \in L(\mathcal{A}_n)$ iff there are $a_1, \dots, a_k \in \{1, \dots, n\}$ such that the words $a_1 a_2, a_2 a_3, \dots, a_k a_1$ occur infinitely often in $\alpha$.
	
	Let $\mathcal{A}' = (Q', \Sigma_n, q_0', \delta, \Omega)$ be a DRA for $L(\mathcal{A}_n)$. We want to show that $\mathcal{B}$ has at least $n!$ states. A permutation word is a word $(i_1 \dots i_n \#)^\omega$ with $\{i_1, \dots, i_n\} = \{1, \dots, n\}$; let $\text{PW}$ be the set of permutation words. Let $\alpha, \beta \in \text{PM}$ with $\alpha \neq \beta$ and let $\rho_\alpha, \rho_\beta$ the respective runs of $\mathcal{B}$. We claim that $\text{Inf}(\rho_\alpha) \cap \text{Inf}(\rho_\beta) = \emptyset$. If that is true, then $|Q'| \geq |\text{PM}| = n!$.
	
	\paragraph{Claim}: $\text{Inf}(\rho_\alpha) \cap \text{Inf}(\rho_\beta) = \emptyset$
	
	Let $\alpha = (\overline{i} \#)^\omega$ and $\beta = (\overline{j} \#)^\omega$ with $\overline{i} = i_1 \dots i_n$ and $\overline{j} = j_1 \dots j_n$. Assume there is a $q \in \text{Inf}(\rho_\alpha) \cap \text{Inf}(\rho_\beta)$. We extract the following segments from the two runs:
	\begin{itemize}
		\item $\rho_1$ from $\rho_\alpha$ with $\rho_1: q_0 \overset{w}{\rightarrow} q$ for some $w \in \Sigma_n^*$.
		\item $\rho_2$ from $\rho_\alpha$ with $\rho_2: q \overset{w}{\rightarrow} q$ for some $w \in \Sigma_n^* \overline{i} \Sigma_n^*$.
		\item $\rho_3$ from $\rho_\beta$ with $\rho_3: q \overset{w}{\rightarrow} q$ for some $w \in \Sigma_n^* \overline{j} \Sigma_n^*$.
		\item $\pi := \rho_1 (\rho_2 \rho_3)^\omega$
	\end{itemize}
	Then $\pi$ is a run of $\mathcal{A}'$ on a word $\gamma$ with $\text{Inf}(\pi) = \text{Inf}(\rho_\alpha) \cup \text{Inf}(\rho_\beta)$. Since $\overline{i} \neq \overline{j}$, there is a minimal $k$ with $i_k \neq j_k$. By this choice, there must also be $l, l'$ such that $i_k = j_l$ and $j_k = i_{l'}$. Therefore, the following pairs all occur infinitely often in $\gamma$: 
	
	$$ i_k i_{k+1}, i_{k+1} i_{k+2}, \dots, i_{l'-1} \underset{=j_k}{i_{l'}}, j_k j_{k+1}, \dots, j_{l-1} \underset{=i_k}{j_l} $$
	
	As shown before, that means $\gamma \in L(\mathcal{A}_n)$, so $\pi$ is an accepting run. This contradicts the Union Lemma (theorem \ref{union_lemma}).
\end{proof}


\vspace{.5cm}
\begin{theorem}
	Given an ABA $\mathcal{A}$, the dual $\tilde{\mathcal{A}}$ is an alternating co-Büchi automaton which accepts $L(\mathcal{A})^\complement$, with $\tilde{F} = Q \setminus F$ and $\tilde{\delta}$ exchanging true/false and $\land$/$\lor$.
\end{theorem}
\begin{proof}
	We claim that $L(\mathcal{A})^\complement = L(\tilde{\mathcal{A}})$.
	
	\paragraph{$\bm{\subseteq}$} Let $\alpha \notin L(\mathcal{A})$, so for every run tree $t$ of $\mathcal{A}$ on $\alpha$, there is a path $\pi_t$ which is rejecting. One can show that the combination of all these paths is an accepting run of $\tilde{\mathcal{A}}$.
	
	\paragraph{$\bm{\supseteq}$} Let $\alpha \in L(\mathcal{A})$, so there is an accepting run tree $t$. Let $\tilde{t}$ be a run tree of $\tilde{\mathcal{A}}$ on $\alpha$. One can show (by induction on the path length) that there is a path in $t$ that is also a path in $\tilde{t}$. Since that path is accepting in $t$, it must be rejecting in $\tilde{t}$.
\end{proof}


\subsection{Minimization}
\begin{theorem}
	There is a DBA-recog. language which does not have a unique minimal DBA. \\
	DBAs minimized with the DFA minimization algorithm can be arbitrarily bad compared to a minimal DBA.
\end{theorem}
\begin{proof}
	The language $(a^* b)^\omega$ has two non-isomorphic DBAs.
	
	For every $n>0$, consider $L_n = \{a^n\}$. The minimal NFA for $L_n$ has $n+1$ states but $\text{lim}(L_n) = \{a^\omega\}$ can be recognized by a DBA with a single state.
\end{proof}


\vspace{.5cm}
\begin{theorem}
	Let $U \subseteq \Sigma^\omega$ be regular and $\sim_U$ be the Myhill-Nerode equivalence relation. Every DBA for $U$ has at least $\text{index}(\sim_U)$ many states.
\end{theorem}
\begin{proof}
	Assume $\mathcal{A} = (Q, \Sigma, q_0, \delta, F)$ is a DBA for $U$ with $|Q| < \text{index}(\sim_U)$. That means there are $u, v \in \Sigma^*$ with $u \not\sim_U v$ but $\delta^*(q_0, u) = \delta^*(q_0, v)$. By definition of the automaton, that means $u \sim_U v$, which is a contradiction.
\end{proof}


\vspace{.5cm}
\begin{theorem}
	For every $n$, there is a $U_n \subseteq \Sigma^\omega$ such that $\text{index}(\sim_{U_n}) = 1$ but every DBA for $U_n$ has at least $n$ states.
\end{theorem}
\begin{proof}
	Let $U_n = \{\alpha \mid a^n b \text{ appears infinitely often as a substring in } \alpha \}$. The language is prefix independent, so $\text{index}(\sim_{U_n}) = 1$. However, a DBA for $U_n$ clearly requires $n+1$ states.
\end{proof}


\vspace{.5cm}
\begin{theorem}
	The problem $\textsc{DBAMin} = \{ (\mathcal{A}, k) \mid \mathcal{A} \text{ DBA}, k \in \mathbb{N}, \text{There is a } k \text{-state DBA for } L(\mathcal{A}) \}$ is NP-complete.
\end{theorem}
% no proof


\vspace{.5cm}
\begin{definition}
	For $U \subseteq \Sigma^\omega$, we define the canonical automaton $\mathcal{A}_U = (Q_U, \Sigma, q_0^U, \delta_U, F_U)$ with 
	\begin{itemize}
		\item $Q_U = \{ [u]_{\sim_U} \mid u \in \Sigma^* \}$
		\item $q_0^U = [\varepsilon]_{\sim_U}$
		\item $\delta_U([u]_{\sim_U}, a) = [ua]_{\sim_U}$
		\item $F_U = \{ [u]_{\sim_U} \mid \exists v \in \Sigma^+: u \sim_U uv \text{ and } uv^\omega \in U\}$ 
	\end{itemize}
\end{definition}


\vspace{.5cm}
\begin{theorem}
	Let $U$ be WDBA-recognizable. Then $\mathcal{A}$ is a WDBA and $L(\mathcal{A}_U) = U$.
\end{theorem}
\begin{proof}
	We show the statement in two steps.

	\paragraph{Claim}: $\mathcal{A}_U$ is a WDBA.
	
	Let $[u]_{\sim_U}, [v]_{\sim_U} \in Q_U$ be two states in the same SCC and $[u]_{\sim_U} \in F_U$.
	 We need to show that $[v]_{\sim_U} \in F_U$ as well. Since the two states are in the same SCC, let $\mathcal{A}_U : [u]_{\sim_U} \overset{y}{\rightarrow} [v]_{\sim_U} \overset{z}{\rightarrow} [u]_{\sim_U}$. We have $u(yz)^\omega \in U$. %TODO F20 notes why??
	 
	 Hence, $uy(zy)^\omega \in U$. By definition we know $[uy]_{\sim_U} = [u']_{\sim_U}$, so $u'(zy)^\omega in U$ which means that $[u']_{\sim_U} \in F_U$.
	 
	 \paragraph{Claim}: $L(\mathcal{A}_U) = U$.
	 
	 Let $uv^\omega$ be an ultimately periodic word. By the pigeon hole principle, there are $m, n > 0$ such that $uv^m \sim_U uv^mv^n$. 
	 
	 \begin{align*}
	 	& u v^\omega \in U \\
	 	\Leftrightarrow & u v^m (v^n)^\omega \in U \\
	 	\Leftrightarrow & [u v^m]_{\sim_U} \in F_U \\
	 	\overset{uv^m \sim_U uv^mv^n}{\Leftrightarrow} & u v^m (v^n)^\omega \in L(\mathcal{A}_U) \\
	 	\Leftrightarrow uv^\omega \in L(\mathcal{A}_U)
	 \end{align*}
	 
	 Now consider $X = (U \cup L(\mathcal{A}_U)) \setminus (U \cap L(\mathcal{A}_U)$. If the two languages are equal, then $X$ must be empty. As we have shown above, $U$ and $L(\mathcal{A}_U)$ contain the same ultimately periodic words, so $X$ does not contain any. It is also constructed by Boolean operations on regular languages, so $X$ must be regular. That means that $X = \emptyset$.
\end{proof}

\vspace{.5cm}
\begin{definition}
	Let $\mathcal{A} = (Q, \Sigma, q_0, \Delta, F)$ be an NBA. A \textbf{looping state} is a state in $Q$ from which there is a non-empty path that ends in that same state.
	
	We define the \textbf{acceptance height} $\text{ah} : Q \rightarrow \mathbb{N}$ as follows: Let $q \in Q$. If all looping states that are reachable from $q$ are accepting, then $\text{ah}(q) := 0$. Otherwise, let $n$ be the maximal acceptance height of a state that is reachable from $q$ but not in the same SCC. If $q$ is non-loopinig, set $\text{ah}(q) := n$. Otherwise, 
	
	$$ \text{ah}(q) := \begin{cases}
		2 \cdot \lfloor \frac{n-1}{2} \rfloor + 2  & \text{if } q \in F \\
		2 \cdot \lfloor \frac{n}{2} \rfloor + 1 & \text{if } q \notin F
	\end{cases} $$
	
	$\mathcal{A}$ is in \textbf{normal form} if $F = \{ q \in Q \mid \text{ah}(q) \text{ is even}\}$.
\end{definition}


\vspace{.5cm}
\begin{theorem}
	Let $\mathcal{A}$ be a weak DBA. One can compute an equivalent weak DBA in normal form in linear time.
\end{theorem}
\begin{proof}
	SCCs can be computed in linear time. Given the SCCs, the inductive definition of ah can be computed in linear time.
\end{proof}


\vspace{.5cm}
\begin{theorem}
	Let $\mathcal{A}$ be a weak DBA in normal form. Minimizing $\mathcal{A}$ as a DFA also results in a minimal weak DBA.
\end{theorem}
% no proof



\subsection{Simulation Game}
\begin{definition}
	Let $\mathcal{A} = (Q, \Sigma, q_0, \Delta, F)$ be an NBA. We define the delayed simulation game $\mathcal{G}_\mathcal{A}(G_\mathcal{A}, \text{Win})$ as follows
	\begin{itemize}
		\item $G_\mathcal{A} = (V_0, V_1, E, c)$ 
		\item $V_0 = \Sigma \times Q \times Q$
		\item $V_1 = Q \times Q$
		\item Player 0 moves from $(a, p, q)$ to a $(p, p')$ with $(q, a, p') \in \Delta$
		\item Player 1 moves from $(q, q')$ to a $(a, p, q')$ with $(q, a, p) \in \Delta$
		\item $c : V \rightarrow \{-1, 0, 1\}$ with $c(v) = \begin{cases} -1 & \text{if } v \in F \times (Q \setminus F) \\ 1 & \text{if } v \in Q \times F \\ 0 & \text{otherwise} \end{cases}$
		\item $\alpha \in \text{Win}$ iff after every -1 in $\alpha$, there is a 1 later on.
	\end{itemize}
	
	Write $q \preceq_\text{de} q'$ if player 0 has a winning strategy from $(q, q')$.
\end{definition}

Idea: Player 1 chooses symbols in $\Sigma$ and transitions on the second state. Player 0 has to answer with transition on the first state which lead to a run of the same acceptance.

\vspace{.5cm}
\begin{theorem}
	If $q \preceq_\text{de} q'$ and $q' \preceq_\text{de} q$, then $q$ and $q'$ can be merged in $\mathcal{A}$ without changing the language of the automaton.
\end{theorem}
% no proof

\vspace{.5cm}
\begin{theorem}
	The delayed simulation game can be reduced to a Büchi game in linear time.
\end{theorem}
 

\newpage 
\section{Finite Trees}
\begin{theorem}[Pumping Principle]
	Let $T \subseteq T_\Sigma$ be a regular ranked tree language. There is a $n \in \mathbb{N}$ such that for all trees $t \in T$, all $m > n$, and all paths $\pi_1 \dots \pi_m$, there are $1 \leq i < j \leq m$ such that for all $k \in \mathbb{N}$:
	$$t[\circ / u] \cdot (t[\circ/v]|_u)^k \cdot t|_v \in T$$
	where $u = \pi_1 \dots \pi_i$ and $v = \pi_1 \dots \pi_j$.
\end{theorem}
% no proof

\begin{definition}
	Let $T \subseteq T_\Sigma$. The \textbf{Myhill-Nerode equivalence} is $\sim_T \subseteq T_\Sigma \times T_\Sigma$ with 
	$$t_1 \sim_T t_2 \Leftrightarrow \forall s \in S_\Sigma: s \cdot t_1 \in T \leftrightarrow s \cdot t_2 \in T$$
	
	The index of $T$ is $\text{Index}(\sim_T) := |T/\sim_T|$.
\end{definition}

\begin{definition}
	Let $T \subseteq T_\Sigma$. We define the canonical DTA $\mathcal{A}_T = (Q_T, \Sigma, \delta_T, F_T)$ as
	\begin{itemize}
		\item $Q_T = \{ [t]_{\sim_T} \mid t \in T_\Sigma \}$.
		\item For all $a \in \Sigma_i$: $\delta_T([t_1]_{\sim_T}, \dots, [t_i]_{\sim_T}, a) = [a(t_1, \dots, t_i)]_{\sim_T}$.
		\item $F_T = \{ [t]_{\sim_T} \mid t \in T \}$.
	\end{itemize}
\end{definition}

\begin{theorem}
	Let $T \subseteq T_\Sigma$. $T$ is regular iff $\text{Index}(\sim_T)$ is finite. If $T$ is regular, $\mathcal{A}_T$ is the minimal DTA.
\end{theorem}
\begin{proof}
	via induction on $t$, prove $\delta_T^*(t) = [t]$
\end{proof}

\begin{theorem}
	The emptiness problem for NTAs can be reduced to \textsf{HORN-SAT} in linear time.
\end{theorem}
\begin{proof}
	Let $\mathcal{A} = (Q, \Sigma, \Delta, F)$ be an NTA. For every $\tau = (q_1, \dots, q_i, a, p) \in \Delta$, let \linebreak $\psi_\tau = (X_{q_1} \land \dots \land X_{q_i} \rightarrow X_q)$. Then we define $\varphi = \bigwedge\limits_{\tau \in \Delta} \psi_\tau \land \bigwedge\limits_{q \in F} X_q \rightarrow 0$. $\varphi$ is satisfiable iff $L(\mathcal{A}) \neq \emptyset$.
\end{proof}

\subsection{BTTs}
\begin{theorem}
	The equivalence problem for BTTs is undecidable.
\end{theorem}
\begin{proof}
	Follows from the equivalence problem of rational word transducer.
\end{proof}

\begin{theorem}
	The emptiness problem for BTTs is decidable in polynomial time.
\end{theorem}
\begin{proof}
	One can construct in polynomial time a $\uparrow$NTA that recognizes the domain of a given BTT. The emptiness problem of BTTs then is equivalent to the emptiness problem of $\uparrow$NTAs.
\end{proof}

\begin{theorem}
	The type-checking problem (given regular $T, T'$, is $\mathcal{A}(T) \subseteq T'$?) is decidable.
\end{theorem}
\begin{proof}
	$\mathcal{A}(T) \subseteq T'$ iff $\mathcal{A}^{-1}((T')^\complement) \cap T = \emptyset$. If $T'$ is regular then so is $\mathcal{A}^{-1}((T')^\complement) \cap T$ and this can be decided.
\end{proof}

\begin{theorem}
	If $T$ is regular, then $\mathcal{A}^{-1}(T)$ is regular. \\
	If $\mathcal{A}$ is linear, then $\mathcal{A}(T_\Sigma)$ is regular.
\end{theorem}
%no proof

\begin{theorem}
	There are BTT-definable relations $R_1, R_2$ such that $R_1 \circ R_2$ is not BTT-definable.
\end{theorem}
\begin{proof}
	Let $\Sigma = \{f,g,h,c\}$ and for both $i \in \{1,2\}$: $\mathcal{A}_i = (\{q,q_f\}, \Sigma, \Sigma, \Delta_i, \{q_f\})$ where 
	\begin{itemize}
		\item $\Delta_1$:
		\begin{itemize}
			\item $c \rightarrow q(c)$
			\item $g(q(x_1)) \rightarrow q(g(x_1))$
			\item $q(x_1) \rightarrow q_f(f(x_1))$
		\end{itemize}
		\item $\Delta_2$:
		\begin{itemize}
			\item $c \rightarrow q(c)$
			\item $g(q(x_1)) \rightarrow q(g(x_1))$
			\item $g(q(x_1)) \rightarrow q(h(x_1))$
			\item $f(q(x_1), q(x_2)) \rightarrow q_f(f(x_1, x_2))$
		\end{itemize}
	\end{itemize}
	
	$\mathcal{A}_1$ converts a tree $t \in T_\{c,g\}$ to $f(t, t)$.
	$\mathcal{A}_2$ replaces an arbitrary number of $g$s by $h$s.
	$R_1 \circ R_2$ defines the set of trees $f(t_1, t_2)$ where $t_1$ and $t_2$ are of equal height.
\end{proof}

\begin{theorem}
	If $\mathcal{A}_1$ is linear \textbf{or} $\mathcal{A}_2$ is deterministic Then $R(\mathcal{A}_1) \circ R(\mathcal{A}_2)$ is BTT-definable.
\end{theorem}
%no proof


\section{Infinite Trees}
\begin{theorem}[BTA Pumping]
	For $t \in T_\Sigma, x \in \{0,1\}^*, y \in \{0,1\}^+$, let 
	$$t^*_{[x,y]} : \{0,1\}^* \rightarrow \Sigma, z \mapsto \begin{cases} t(z) & \text{if } xy \not\sqsubseteq z \\ xz' & \text{if } \exists n>0: z = xy^nz' \text{ with } y \not\sqsubseteq z' \end{cases}.$$
	Let $\mathcal{A}$ be a BTA, $t \in T(\mathcal{A})$, $\rho$ an accepting run of $\mathcal{A}$ on $t$, and $x, y, y' \in \{0,1\}^*$ s.t. $\rho(x) = \rho(xy)$, $y' \sqsubset y$, and $\rho(xy') \in F$. Then $t^*_{[x,y]} \in T(\mathcal{A})$.
\end{theorem}
\begin{proof}
	$\rho^*_{[x,y]}$ is an accepting run of $\mathcal{A}$ on $t^*_{[x,y]}$.
\end{proof}

\begin{theorem}
	Every non-empty regular tree language contains a regular tree.
\end{theorem}
\begin{proof}
	Let $\mathcal{A}$ be a PTA for $T = T(\mathcal{A}) \neq \emptyset$ and let $\mathcal{G}_\mathcal{A}$ be the emptiness game for $\mathcal{A}$. Since $T$ is not empty, player Automaton has a positional winning strategy $\sigma : Q \rightarrow \Sigma$. For $\sigma(q) = (q, a, q_0, q_1)$, we define $\delta_\sigma(q, 0) = q_0$, $\delta_\sigma(q, 1) = q_1$, and $f_\sigma(q) = a$. Then $t \in T$ where $t$ is the regular tree defined by $\mathcal{B} = (Q, \{0,1\}, q_0, \delta_\sigma, f_\sigma)$.
\end{proof}

\begin{theorem}[Rabin's Tree Theorem]
	The MSO theory of $\underline{T_2}$ is decidable for formulas $\varphi(X_1, \dots, X_n)$ and a model $X_1, \dots X_n \subseteq \{0,1\}^*$ is computable.
\end{theorem}
\begin{proof}
	Transform $\varphi$ into an equivalent PTA. A model can be found by solving the emptiness game.
\end{proof}


\end{document}
















