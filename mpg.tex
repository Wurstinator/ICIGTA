\documentclass{article}

\usepackage{amsmath, amssymb, amsthm}
\usepackage[utf8]{inputenc}
\usepackage{geometry}
\newtheorem{theorem}{Theorem}
\newtheorem{definition}{Definition}

\begin{document}

\section{Mean Payoff Games}
\begin{definition}
	A \textbf{mean payoff game} is a two player-game with the arena $\mathcal{G} = (V_0, V_1, E, r)$, where $(V_0, V_1, E)$ is a game graph and $r : E \rightarrow \mathbb{Z}$. 
	
	The value of a finite play is the average over all edges $r_\text{fin}(v_0 \dots v_n) = \frac{1}{n} \cdot \sum\limits_{i=0}^{n-1} r(v_i, v_{i+1})$.  \\
	
	The infinite variant has plays $\pi \in V^\omega$. The goal of player 0 is to maximize $r_0(\pi) = \underset{n \rightarrow \infty}{\text{lim inf }} r_\text{fin}(\pi[0,n])$, whereas the goal of player 1 is to minimize $r_1(\pi) = \underset{n \rightarrow \infty}{\text{lim sup }} r_\text{fin}(\pi[0,n])$. \\
	The best values the players can reach from $v \in V$ are $\text{val}_0(v) := \sup\limits_\sigma \inf\limits_\tau r_0(\pi_{\sigma,\tau,v})$ and $\text{val}_0(v) := \inf\limits_\tau \sup\limits_\sigma r_1(\pi_{\sigma,\tau,v})$.
	
	The finite variant $\mathcal{G}_\text{fin}$ has plays $\pi = v_0 \dots v_n$ for $n \leq |V|+1$ that terminate as soon as a vertex is reached the second time, i.e. $v_n = v_i$ for some $i < n$. The value is $r_\text{fin}(v_i \dots v_n)$. \\
	
	The finite variant with resets at $u \in V$  ($\mathcal{G}_\text{fin}^u$) has plays $\pi = v_0 \dots v_n$ for $n \leq 2|V| + 1$. The game behaves like the finite variant, with the addition that the \textbf{first} occurrence of $u$ in a $\pi$ resets the game, i.e. $v_1 v_3 u v_1 v_2 u$ is a valid run with score $r_\text{fin}(u v_1 v_2 u)$.
\end{definition}

\begin{definition}
	Let $\mathcal{G} = (V_0, V_1, E, r)$ be a mean payoff game. Let $v \in V$ and let $\sigma$ and $\tau$ be strategies of player 0 and 1 respectively. The unique run of $\sigma$ and $\tau$ from $v$ is $\pi_{\sigma,\tau,v} \in V^\omega$ for the infinite variant or $\beta_{\sigma,\tau,v} \in V^*$ for the finite variant.
\end{definition}

\section{Determinacy}
\begin{theorem}
	Let $\mathcal{G} = (V_0, V_1, E, r)$ be a mean payoff game. Then the best value either player can achieve in the finite variant for any $v \in V$ is $\max\limits_\sigma \min\limits_\tau r_\text{fin}(\beta_{\sigma,\tau,v}) = \min\limits_\tau \max\limits_\sigma r_\text{fin}(\beta_{\sigma,\tau,v})$. We call this value $\text{val}(v, \mathcal{G}_\text{fin})$. 
\end{theorem}
\begin{proof}
	The proof is based on the Minmax theorem bei von Neumann. Given a finite tree of all possible plays in the game, one can derive the value at each node from the leaf upwards. Note that these strategies are not positional in general.
\end{proof}

The same proof can be used to find a unique $\text{val}(v, \mathcal{G}^u_\text{fin})$ for the finite variant with resets,

\begin{definition}
	Let $\mathcal{G} = (V_0, V_1, E, r)$ be a mean payoff game. Let $\sigma$ be a strategy for player 0/1 in $\mathcal{G}$ or $\mathcal{G}_\text{fin}$. $\sigma$ secures $a \in \mathbb{R}$ for that player if every play according to $\sigma$ has value at least / most $a$.
\end{definition}

\begin{theorem}
\label{thm:sec_carry}
	Let $\mathcal{G} = (V_0, V_1, E, r)$ be a mean payoff game and let $\sigma$ be a strategy for one of the players in $\mathcal{G}_\text{fin}$. We define $\hat{\sigma}(\pi) = \sigma(\pi')$, where $\pi'$ is obtained by removing all loops from $\pi$. If $\sigma$ secures value $a$ in $\mathcal{G}_\text{fin}$ for that player, then $\hat{\sigma}$ secures value $a$ in $\mathcal{G}$ for that player.
\end{theorem}
\begin{proof}
	Let $\pi \in V^\omega$ be a play in $\mathcal{G}$ according to $\hat{\sigma}$. After a finite prefix, $\pi$ consists of loops only (because $V$ is finite). We claim that every one of those loops has a value of at least/most $a$. If that is true, $\pi$ also has a value of at least/most $a$ which is what had to be shown.
	
	Let $v_1 \dots v_k v_1$ be a loop in $\pi$, so $\rho v_1 \dots v_k v_1 \sqsubseteq \pi$ for some $\rho$. We can assume that $\rho v_1 \dots v_k$ does not contain any loops. If there would be a loop containing any $v_i$, then that loop would have to be the same as $v_1 \dots v_k v_1$ by definition of $\hat{\sigma}$, so we can adapt $\rho$ accordingly. Any other loop in $\rho$ could simply be removed. \\
	Thus, $\rho v_1 \dots v_k v_1$ is a play in $\mathcal{G}_\text{fin}$ and moreover it is a play according to $\sigma$. Because $\sigma$ secures value $a$, the loop $v_1 \dots v_k v_1$ must have value at least/most $a$.
\end{proof}

\begin{theorem}
\label{thm:inf_determined}
	Let $\mathcal{G} = (V_0, V_1, E, r)$ be a mean payoff game. Then $\text{val}_0(v) = \text{val}_1(v) = \text{val}(v, \mathcal{G}_\text{fin})$ for all $v \in V$. We call this value $\text{val}(v, \mathcal{G})$.
\end{theorem}
\begin{proof}
	Let $a := \text{val}(v, \mathcal{G}_\text{fin})$, so there are strategies $\sigma$ and $\tau$ for player 0 and 1 respectively that both secure $a$ in $\mathcal{G}_\text{fin}$. By theorem \ref{thm:sec_carry}, the same is true for $\mathcal{G}$, so $\text{val}_1(v) \leq a \leq \text{val}_0(v)$. 
	
	From the definition of $\text{val}_0$ and $\text{val}_1$ we also have $\text{val}_0(v) \leq \text{val}_1(v)$. The two inequations imply our goal.
\end{proof}

\section{Positionality}
\begin{theorem}
\label{thm:fin_ufin_eq}
	For all $u, v \in V$: $\text{val}(v, \mathcal{G}_\text{fin}) = \text{val}(v, \mathcal{G}^u_\text{fin})$.
\end{theorem}
\begin{proof}
	We show that $a := \text{val}(v, \mathcal{G}_\text{fin}) \leq \text{val}(v, \mathcal{G}^u_\text{fin})$. The other bound and equality can be shown symetrically. Let $\sigma$ be an optimal strategy for player 0 in $\mathcal{G}_\text{fin}$. If there are no plays from $v$ according to $\sigma$ that visit $u$, then $\sigma$ secures $a$ for player 0 in $\mathcal{G}^u_\text{fin}$. Otherwise:

	\textbf{Claim 1}: $a \leq \text{val}(u, \mathcal{G}^u_\text{fin})$.
	
	By theorem \ref{thm:inf_determined}, we know that $\text{val}(v, \mathcal{G}) = \text{val}(v, \mathcal{G}_\text{fin}) = a$. Let $\tau$ be an arbitrary strategy for player 1 and let $\tau'$ be the strategy that plays from $v$ to $u$ (which is possible by assumption) and then plays according to $\tau$. Note that $\pi_{\hat{\sigma},\tau',v} = v \dots u \pi_{\hat{\sigma},\tau,u}$ and therefore $\text{val}(u, \mathcal{G}_\text{fin}) = \text{val}(u, \mathcal{G}) = \text{val}(v, \mathcal{G}) = a$.
	
	It remains to be shown that $\text{val}(u, \mathcal{G}_\text{fin}) \leq \text{val}(u, \mathcal{G}^u_\text{fin})$. This is clear because the ''reset`` of every run happens at the initial vertex. 
	%TODO does this proof make sense?
\end{proof}

\begin{theorem}
	Let $\mathcal{G}$ be a parity game. Both players have positional optimal strategies for $\mathcal{G}_\text{fin}$, i.e. there are strategies $\sigma^*$ and $\tau^*$ for player 0 and 1 respectively such that $\text{val}(v, \mathcal{G}_\text{fin}) = r_\text{fin}(\beta_{\sigma^*,\tau^*,v})$.
\end{theorem}
\begin{proof}
	We perform a proof by induction on $|E|$. If $|E| = |V|$, then there is only one choice at every vertex and positional optimal strategies are easy to define.
	
	Otherwise, there is some node $u \in V$ which has at least two outgoing edges. By theorem \ref{thm:fin_ufin_eq}, $\text{val}(v, \mathcal{G}_\text{fin}) = \text{val}(v, \mathcal{G}^u_\text{fin})$ for all $v$. Let $\text{val}(v, \mathcal{G}_\text{fin}) = a$, then there is a strategy $\sigma$ which secures value $a$ in $\mathcal{G}^u_\text{fin}$. 
	
	Let $w \in V$ be arbitrary such that there is a play $\rho$ which leads to $\sigma(u) = w$. Note that we can assume this $w$ to be unique, as any history before a first occurrence of $u$ has no effect on the outcome and a second occurrence terminates the run. We then obtain $\mathcal{G}'$ from $\mathcal{G}$ by removing all outgoing edges from $u$ except for $(u, w)$.
	
	By induction, there is a positional optimal strategy $\sigma'$ for $\mathcal{G}'_\text{fin}$ which secures value $a' = \text{val}(v, \mathcal{G}'_\text{fin})$ for the player. Using the same strategy $\sigma'$ for $\mathcal{G}_\text{fin}$ must therefore also secure $a'$. It remains to be shown that $a = a'$.
	
	By theorem \ref{thm:fin_ufin_eq}, we have $a' = \text{val}(v, \mathcal{G}_\text{fin}^{\prime u})$ and therefore $a' = \text{val}(v, \mathcal{G}_\text{fin}^u) = a$ by choice of $\sigma'$.
\end{proof}

\begin{theorem}
	Let $\mathcal{G}$ be a parity game. Both players have positional optimal strategies for $\mathcal{G}$, i.e. there are strategies $\sigma^*$ and $\tau^*$ for player 0 and 1 respectively such that $\text{val}(v, \mathcal{G}) = r_0(\pi_{\sigma^*,\tau^*,v}) = r_1(\pi_{\sigma^*,\tau^*,v})$.
\end{theorem}
\begin{proof}
	Using theorem \ref{thm:sec_carry} and \ref{thm:inf_determined}, we know that $\hat{\sigma^*}$ and $\hat{\tau^*}$ both secure $\text{val}(v, \mathcal{G})$ for their respective players. They are both positional by definition.
\end{proof}

\section{Algorithm}
\begin{theorem}
\label{thm:alg_vals}
	Let $\mathcal{G}$ be a parity game. For every $v \in V$ the value $\text{val}(v, \mathcal{G})$ can be computed in $\text{poly}(W + |V|)$ time, where $W = \max \{ |x| \mid x \in r(V) \}$.
\end{theorem}
\begin{proof}
\textbf{Claim 1}: For $|V| = n$, the difference between two possible values for a $\text{val}(v)$ is at least $\frac{1}{n(n-1)}$. 

Let $x, y \in \mathbb{R}$ be values of a vertex from any mean payoff game with $|V| = n$. Because of the positional determinacy of the game, the value must be defined by the average of some loop in the arena, so $x = \frac{k_1}{l_1}, y = \frac{k_2}{l_2}$ for some $k_1, k_2 \in \mathbb{Z}$ and $l_1, l_2 \in \mathbb{N}$ with $l_1, l_2 \leq n$. Our claim is: $|\frac{k_1}{l_1} - \frac{k_2}{l_2}| \geq \frac{1}{n(n-1)}$.

If $l_1 = l_2$, then $k_1 \neq k_2$ and $|\frac{k_1}{l_1} - \frac{k_2}{l_2}| = |\frac{k_1 - k_2}{l_1}| \geq \frac{1}{l_1} \geq \frac{1}{n} \geq \frac{1}{n(n-1)}$.

If $l_1 \neq l_2$, then $|\frac{k_1}{l_1} - \frac{k_2}{l_2}| = |\frac{k_1 l_2 - k_2 l_1}{l_1 l_2}| \geq |\frac{k_1 l_2 - k_2 l_1}{n (n-1)}| \geq \frac{1}{n (n-1)}$.

\textbf{Claim 2}: Let $\text{val}^k(v, \mathcal{G})$ be the optimal mean value of plays of length $k$ that can be enforced, i.e. the best score if every run is terminated after $k$ steps. Then $\text{val}^k$ can be computed in $\mathcal{O}(k \cdot |E|)$.

We have $\text{val}^0(v, \mathcal{G}) = 0$ for all $v$. Assuming that $\text{val}^k$ was already defined, we can compute $\text{val}^{k+1}$ in $\mathcal{O}(|E|)$ as follows:
$$\text{val}^{k+1}(v, \mathcal{G}) = \begin{cases}
	\max\limits_{(u,v) \in E} \frac{k \cdot \text{val}^k(v, \mathcal{G}) + r(u, v)}{k+1} & \text{if } v \in V_0 \\
	\min\limits_{(u,v) \in E} \frac{k \cdot \text{val}^k(v, \mathcal{G}) + r(u, v)}{k+1} & \text{if } v \in V_1
\end{cases}$$

\textbf{Claim 3}: For all $v \in V, k \in \mathbb{N}$, we have $|\text{val}^k(v) - \text{val}(v)| \leq \frac{2nW}{k}$. % no proof

\textbf{Claim 4}: $\text{val}(v, \mathcal{G})$ can be computed in $\text{poly}(W + |V|)$ time.

Let $k = 4n^3W$. Then $|\text{val}^k(v) - \text{val}(v)| \leq \frac{2nW}{4n^3W} = \frac{1}{2 n^2} \leq \frac{1}{2n(n-1)}$, so there is exactly one possible value for $\text{val}(v)$ in the neighborhood of $\text{val}^k(v)$. This can be computed by checking which interval $[(\text{val}^k(v) - \frac{1}{2n(n-1)}) \cdot l, (\text{val}^k(v) + \frac{1}{2n(n-1)}) \cdot l]$ for $0 \leq l \leq n$ contains an integer $i$. Then $\text{val}(v) = \frac{i}{l}$.
\end{proof}

\begin{theorem}
	Let $\mathcal{G}$ be a parity game. There is an algorithm to construct the positional optimal strategies $\sigma^*$ and $\tau^*$ in $\text{poly}(W + |V|)$ time, where $W = \max \{ |x| \mid x \in r(V) \}$.
\end{theorem}
\begin{proof}
	We construct the strategy $\sigma$ for the player step by step, i.e. for every $v \in V_0$ we define a successor $\sigma(v)$.
	
	If $v$ has exactly one successor, the choice of $\sigma(v)$ is clear. Otherwise, let $u$ be an arbitrary successor of $v$. We define the game $\mathcal{G}'$ by removing the edge $(v, u)$ from $\mathcal{G}$. Using theorem \ref{thm:alg_vals}, we can compute $a := \text{val}(v, \mathcal{G})$ and $b := \text{val}(v, \mathcal{G}')$ in polynomial time.
	
	If $a = b$, then the edge $(v, u)$ is not a part of $\sigma^*$, so we can continue the algorithm with $\mathcal{G}'$. If $a \neq b$, then define $\sigma(v) = u$.
\end{proof}

\section{Connection to Parity Games}
\begin{theorem}
	Let $\mathcal{G}$ be a parity game. We can construct a mean payoff game $\mathcal{G}'$ over the same game graph such that $\text{val}(v, \mathcal{G}) \geq 0$ iff player 0 has a winning strategy in $\mathcal{G}'$ from $v$.
\end{theorem}
\begin{proof}
	Let $\mathcal{G} = (V_0, V_1, E, c)$ with $|V| = n$. We define $\mathcal{G}' = (V_0, V_1, E, r)$ with $r(u, v) = (-n)^{c(u)}$. We claim that $\text{val}(v, \mathcal{G}') \geq 0$ iff player 0 has a winning strategy in $\mathcal{G}$ from $v$. Due to symmetry, we only show one direction.
	
	Let $\overline{\sigma}$ be a positional winning strategy of player 0 in $\mathcal{G}$ from $v$. Let $\tau^*$ be a positional optimal strategy for player 1 in $\mathcal{G}'$. We have $\text{val}(v, \mathcal{G}') = \text{val}_1(v, \mathcal{G}') = \sup\limits_\sigma r_1(\pi_{\sigma,\tau^*,v}) \geq r_1(\pi_{\overline{\sigma},\tau^*,v})$. It remains to be shown that $r_1(\pi_{\overline{\sigma},\tau^*,v}) \geq 0$.
	
	Because $\overline{\sigma}$ and $\tau^*$ are both positional, $\pi_{\overline{\sigma},\tau^*,v} = v_0 \dots v_i (v_{i+1} \dots v_k v_i)^\omega$ is ultimately periodic with $k - i + 1 \leq n$. Then $r_1(\pi_{\overline{\sigma},\tau^*,v}) = r_\text{fin}(v_{i+1} \dots v_k v_i)$. Since $\overline{\sigma}$ is a winning strategy for player 0 in $\mathcal{G}$, the maximal priority in that loop must be even. Let that priority be $p$.
	
	$$\text{val}(v, \mathcal{G}') = r_\text{fin}(v_{i+1} \dots v_k v_i) \geq n^p - k \cdot n^{p-1} \geq n^p - (n-1) \cdot n^{p-1} = n^{p-1} > 0$$
\end{proof}

\end{document}
















